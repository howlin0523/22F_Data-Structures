\documentclass[10pt, a4paper, oneside]{ctexart}
\usepackage{amsmath, amsthm, amssymb, bm, color, enumerate, framed, graphicx, hyperref, listings, mathrsfs, xcolor}

\lstset{
	language=c++,
	breaklines,
	columns=fixed,
	numbers=left,                                        % 在左侧显示行号
	numberstyle=\tiny\color{gray},                       % 设定行号格式
	frame=true,                                          % 不显示背景边框
	backgroundcolor=\color[RGB]{245,245,244},            % 设定背景颜色
	keywordstyle=\color[RGB]{40,40,255},                 % 设定关键字颜色
	numberstyle=\footnotesize\color{darkgray},
	commentstyle=\it\color[RGB]{0,96,96},                % 设置代码注释的格式
	stringstyle=\rmfamily\slshape\color[RGB]{128,0,0},   % 设置字符串格式
	showstringspaces=false,                              % 不显示字符串中的空格
   }



\title{\textbf{一元稀疏多项式计算器\ 实验报告}}
\author{howlinggggg}
\date{\today}
\linespread{1.2}

\begin{document}
\maketitle
\section{实验任务}
\subsection{基本要求}
\begin{enumerate}
	\item 输入并建立多项式;
	\item 输出多项式, 输出形式为整数序列; $n, c_1, e_1, c_2, e_2, \dots, c_n, e_n$, 其中 $n$ 是多项式的项数, $c_i$ 和 $e_i$ 分别是第 $i$ 项的系数和指数, 序列按指数降序排列;
	\item 多项式 $a$ 和 $b$ 相加, 建立多项式 $a+b$;
	\item 多项式 $a$ 和 $b$ 相减, 建立多项式 $a-b$.
\end{enumerate}
\subsection{选做内容}
\begin{enumerate}
	\item [1.]计算多项式在 $x$ 处的值。
	\item [2.]求多项式 $a$ 的导函数 $a'$
	\item [4.]多项式的输出形式为类数学表达式.
\end{enumerate}

\section{概要设计}
\begin{figure*}[htbp]
	\centering
	\includegraphics[width = 0.63\textwidth]{latex/main_8cpp_ae66f6b31b5ad750f1fe042a706a4e3d4_cgraph.pdf}
\end{figure*}
\section{详细设计}
\subsection{class Func}
\begin{lstlisting}
	在class定义了此次实验需要用到的函数
class Func
{
private:
    vector<LNode *> head;

public:
    Func()
    {
        for (int i = 0; i < 6; i++)
        {
            LNode *headx = new (LNode);
            headx->coef = 0;
            headx->next = NULL;
            head.push_back(headx);
        }
    }
    bool init();
    void insert(LNode *node, int n);
    void printPoly(int n);
    void add();
    void minus();
    float calValue(float x);
    void derive();
};
\end{lstlisting}
\subsection{bool init()}
初始化多项式a和多项式b,将输入使用insert()插入linklist中
\begin{lstlisting}
bool Func::init()
{
    cout << "input poly a:";
    float t;
    vector<float> temp_a;
    while (cin >> t)
    {
        temp_a.push_back(t);
    }
    if (temp_a.size() % 2 != 0)
    {
        cout << "error" << endl;
        return false;
    }
    for (size_t i = 0; i < temp_a.size(); i += 2)
    {
        LNode *node = new (LNode);
        node->coef = temp_a[i];
        node->expn = temp_a[i + 1];
        insert(node, 1);
    }
    cin.clear();
    cin.ignore();

    cout << "input poly b:";
    vector<float> temp_b;
    while (cin >> t)
    {
        temp_b.push_back(t);
    }
    if (temp_b.size() % 2 != 0)
    {
        cout << "input error" << endl;
        return false;
    }
    for (size_t i = 0; i < temp_b.size(); i += 2)
    {
        LNode *node = new (LNode);
        node->coef = temp_b[i];
        node->expn = temp_b[i + 1];
        insert(node, 2);
    }
    printPoly(1);
    printPoly(2);
    return true;
}
\end{lstlisting}
\subsection{void insert(LNode *node, int n)}
将LNode插入第n个linklist中,其中需要判断是否存在和插入的LNode相同的expn,若存在则将coef相加,否则插入到linklist中并将后面的节点后移。
\begin{lstlisting}
void Func::insert(LNode *node, int n)
{
    LNode *t = head[n - 1];
    while (t->next)
    {
        if (node->expn > t->next->expn)
        {
            node->next = t->next;
            t->next = node;
            ++head[n - 1]->coef;
            return;
        }
        else if (node->expn == t->next->expn)
        {
            t->next->coef += node->coef;
            return;
        }
        t = t->next;
    }
    if (!t->next)
    {
        node->next = t->next;
        t->next = node;
    }
    ++head[n - 1]->coef;
}
\end{lstlisting}
\subsection{void printPoly(int n)}
实现输出多项式和选做内容4, 需要特别处理的是若expn为0,则只输出coef;若coef为$\pm 1$则只输出expn。在选做4中还必须特别考虑第一项,若第一项为+则只输出系数。
\begin{lstlisting}
void printPoly(int n)
{
    LNode *t = head[n - 1];
    LNode *t1 = head[n - 1];
    cout << head[n - 1]->coef;
    while (t->next)
    {
        cout << "," << t->next->coef << "," << t->next->expn;
        t = t->next;
    }
    cout << endl;

    int temp = 0;

    cout << head[n - 1]->coef << ":";
    while (t1->next)
    {
        if (temp != 0 && t1->next->coef > 0)
            cout << "+";
        if (t1->next->coef == -1)
            cout << "-";
        if (t1->next->coef != 1 && t1->next->coef != -1)
            cout << t1->next->coef;
        else if ((t1->next->coef == 1 || t1->next->coef == -1) && t1->next->expn == 0)
            cout << 1;
        if (t1->next->expn != 0 && t1->next->expn != 1)
            cout << "x^" << t1->next->expn;
        else if (t1->next->expn == 1)
            cout << "x";
        t1 = t1->next;
        temp++;
    }
    cout << endl;
}
\end{lstlisting}
\subsection{void plus()}
将多项式相加并插入linklist[2],若相加后系数为0则不插入
\begin{lstlisting}
void Func::plus()
{
    LNode *t1 = head[0];
    LNode *t2 = head[1];
    while (t1->next && t2->next)
    {
        if (t1->next->expn == t2->next->expn)
        {
            LNode *node = new (LNode);
            node->coef = t1->next->coef + t2->next->coef;
            node->expn = t1->next->expn;
            if (node->coef != 0)
                insert(node, 3);
            t1 = t1->next;
            t2 = t2->next;
        }
        else if (t1->next->expn > t2->next->expn)
        {
            LNode *node = new (LNode);
            node->coef = t1->next->coef;
            node->expn = t1->next->expn;
            insert(node, 3);
            t1 = t1->next;
        }
        else
        {
            LNode *node = new (LNode);
            node->coef = t2->next->coef;
            node->expn = t2->next->expn;
            insert(node, 3);
            t2 = t2->next;
        }
    }
    if (!t1->next)
    {
        while (t2->next)
        {
            LNode *node = new (LNode);
            node->coef = t2->next->coef;
            node->expn = t2->next->expn;
            insert(node, 3);
            t2 = t2->next;
        }
    }
    else if (!t2->next)
    {
        while (t1->next)
        {
            LNode *node = new (LNode);
            node->coef = t1->next->coef;
            node->expn = t1->next->expn;
            insert(node, 3);
            t1 = t1->next;
        }
    }
    printPoly(3);
}
\end{lstlisting}

\subsection{void minus()}
将多项式相减并插入linklist[3],若相减后系数为0则不插入
\begin{lstlisting}
void Func::minus()
{
    LNode *t1 = head[0];
    LNode *t2 = head[1];
    while (t1->next && t2->next)
    {
        if (t1->next->expn == t2->next->expn)
        {
            LNode *node = new (LNode);
            node->coef = t1->next->coef - t2->next->coef;
            node->expn = t1->next->expn;
            if (node->coef != 0)
                insert(node, 4);
            t1 = t1->next;
            t2 = t2->next;
        }
        else if (t1->next->expn > t2->next->expn)
        {
            LNode *node = new (LNode);
            node->coef = t1->next->coef;
            node->expn = t1->next->expn;
            insert(node, 4);
            t1 = t1->next;
        }
        else
        {
            LNode *node = new (LNode);
            node->coef = -1 * t2->next->coef;
            node->expn = t2->next->expn;
            insert(node, 4);
            t2 = t2->next;
        }
    }
    if (!t1->next)
    {
        while (t2->next)
        {
            LNode *node = new (LNode);
            node->coef = -1 * t2->next->coef;
            node->expn = t2->next->expn;
            insert(node, 4);
            t2 = t2->next;
        }
    }
    else if (!t2->next)
    {
        while (t1->next)
        {
            LNode *node = new (LNode);
            node->coef = t1->next->coef;
            node->expn = t1->next->expn;
            insert(node, 4);
            t1 = t1->next;
        }
    }
    printPoly(4);
}
\end{lstlisting}
\subsection{float calValue(float x)}
\begin{lstlisting}
float Func::calValue(float x)
{
    float sum = 0;
    LNode *t = head[0];
    while (t->next)
    {
        sum += t->next->coef * pow(x, t->next->expn);
        t = t->next;
    }
    return sum;
}
\end{lstlisting}

\subsection{void derive()}
求得$a'$并存入linklist[4]
\begin{lstlisting}
void Func::derive()
{
    LNode *t = head[0];

    while (t->next)
    {
        LNode *t2 = new (LNode);
        if (t->next->expn != 0)
        {
            t2->coef = t->next->coef * t->next->expn;
            t2->expn = t->next->expn - 1;
            insert(t2, 5);
        }
        t = t->next;
    }
    printPoly(5);
}
\end{lstlisting}
\section{测试结果}
\subsection{输入数据}
多项式a: $\left(2 x+5 x^8-3.1 x^{11}\right)$, 多项式b:$\left(7-5 x^8+11 x^9\right)$
\begin{figure*}[htbp]
	\centering
	\includegraphics*[width = 0.8\textwidth]{image/result.png}
\end{figure*}

\section{实验总结}
在此次实验中熟悉了对单链表的操作
\end{document}
